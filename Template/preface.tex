\section*{前~~~~言}

回想一下,我接触计算机编程已经10年了。

10年前,我报名参加了计算机二级考试,第一次接触计算机编程。

5年前,我考上大学,主修计算机专业,开始系统学习编程技术。和大多数人一样,学到的第一个程序是“Hello, World!”。

\begin{lstlisting}[language=c]
#include <stdio.h>

int main() {
	printf("Hello, World!");
	return 0;
}
\end{lstlisting}

那一年,我18岁,问候了世界。

这5年里,我刷过一些OJ的题目,参加了一些竞赛,深感自己需要一份算法模板文件,参考一些书籍整理了这份资料,包含了常用算法的Python及C++语言描述。

\textbf{注意:}本文使用的语言版本为Python 3.6+及C++ 11;文中所述的数据溢出情况只针对C++,Python无限制但过大的数据不做优化会影响性能。

\section*{致~~~~谢}

本文参考了如下资料,向资料作者表示感谢。

\begin{enumerate}
	\item 胡凡, 曾磊. 算法笔记, 2016.
	\item 刘汝佳. 算法竞赛入门经典(第2版), 2014.
	\item 啊哈磊. 啊哈!算法, 2014.
	\item 严蔚敏, 吴伟民. 数据结构(C语言版), 2009.
	\item Stephen Prata. C++ Primer Plus, 6th ed, 2011.
	\item GitHub项目: \href{https://github.com/kickstartcoding/cheatsheets}{kickstartcoding/cheatsheets}
\end{enumerate}
