\subsection{进制转换}

对于一个$P$进制数$x$,将其转换为$Q$进制数$z$,需要分为两步:

(1)将$x$转换为十进制数$y$,若$P$进制数$x=a_1a_2...a_n$,则$y=a_1*P^{n-1}+a_2*P^{n-2}+...+a_{n-1}*P+a_n$;

\lstinputlisting[style=py]{src/pto10.py}

\lstinputlisting[style=cpp]{src/pto10.cpp}

(2)将$y$转换为$Q$进制数$z$,采用“除基取余法”,每次将待转换数除以$Q$,然后将得到的余数作为低位存储,而商则继续除以$Q$并进行上面的操作,最后当商为0时,将所有位从高到低输出就可以得到$z$。

\lstinputlisting[style=py]{src/10toq.py}

\lstinputlisting[style=cpp]{src/10toq.cpp}

\subsection{递归}

\subsubsection{分治}

分治法将原问题划分成若干个\textbf{规模较小}而\textbf{结构与原问题相同}的子问题,然后分别解决,最后\textbf{合并子问题的解},即可得到原问题的解。

\subsubsection{递归}

递归适合用来实现分治思想。递归函数中包含两个重要的概念:\textbf{递归边界}和\textbf{递归调用}。

\paragraph{Fibonacci数列}

可以使用递归式$F(n)=F(n-1)+F(n-2), F(0)=F(1)=1$编写程序,但会造成重复计算的问题;推荐写递推程序,不会产生爆栈问题。

\begin{lstlisting}
long long fib(int n) {
	if (n <= 1)
		return 1;
	long long a = 1, b = 1;
	for (int i = 2; i <= n; ++i) {
		b = a + b;
		a = b - a;
	}
	return b;
}
\end{lstlisting}

\paragraph{全排列}

对于全排列,可以使用深度优先搜索实现,也可使用STL next\_permutation函数,效率更高。

\begin{lstlisting}[language=c++]
// 函数原型
bool next_permutation(begin, end[, cmp]]);
// 示例程序
int perm[6] = {1, 2, 3, 4, 5, 6}, count = 0;
do {
	++count;
} while(next_permutation(perm, perm + 6));
\end{lstlisting}

也可使用递归程序实现

\begin{lstlisting}[language=c++]
int p[maxn];
bool hashTable[maxn] = {false};
void generateP(int index, int n) {
	if (index == n + 1) {
		// 递归边界
		return;
	}
	for(int x = 1; x <= n; x++) {
		if (hashTable[x] == false) {
			p[index] = x;
			hashTable[x] = true;
			generateP(index + 1, n);
			hashTable[x] = false;
		}
	}
}
\end{lstlisting}

\subsubsection{回溯}

在到达递归边界的某一层,由于一些事实导致已经不需要任何一个子问题的递归,就可以直接返回上一层。这种做法称为\textbf{回溯法}。

\paragraph{$n$皇后问题}

在$n*n$的国际象棋棋盘上放置$n$个皇后,两两不在同一行、同一列、同一条对角线上,求合法方案数。

可以使用全排列来实现。当放置部分皇后时,可能剩余的皇后无论怎样放置都不可能合法,此时就没必要递归了,直接返回上一层即可,这样可以减少很多计算量。

\begin{lstlisting}
int n, count = 0, p[maxn];
bool hashTable[maxn] = {false};
void generateP(int index) {
	if (index == n + 1) {
		count++;
		return;
	}
	for (int x = 1; x <= n; x++) {
		if (hashTable[x] == false) {
			bool flag = true;
			for (int pre=1; pre  <index; pre++) {
				if(abs(index-pre)==abs(x-p[pre])){
					flag = false;
					break;	// 与之前皇后同一对角线
				}
			}
			if (flag) {
				p[index] = x;
				hashTable[x] = true;
				generateP(index + 1);
				hashTable[x] = false;
			}
		}
	}
}
\end{lstlisting}

\subsection{排序}

\subsubsection{STL sort}

STL中包含一些进行排序操作的函数,函数的操作区间为[begin, end),使用cmpfunc指定二元比较函数,默认的比较函数是operator<。这里的begin、middle、end都是迭代器iterator。

\begin{lstlisting}
// 使用随机化快速排序对区间内所有元素排序,不稳定
void sort(begin, end[, cmpfunc]);

// 若 m 是排序后的第 middle-begin 个元素
// 则将 m 放置在 middle 位置
// 排在 m 前后的两个区间都是无序的
void nth_element(begin, middle, end[, cmp]);

// 判断一个区间是否有序
bool is_sorted(begin, end[, cmp]);

// 符合一元判断函数test条件的移到前面,不符的移到后面
// 返回指向第一个不符合条件元素的迭代器 mid
// 符合的在[begin,end)之间,不符合的在[mid,end)之间
iterator partition(begin, end, func test);
// 稳定的partition
iterator stable_partition(begin, end,func test);
\end{lstlisting}

\subsubsection{快速排序}

快速排序是冒泡排序的一种改进,使用分治思想,时间复杂度为$O(n\log n)\sim O(n^2)$,空间复杂度为$O(\log n)$。

将第一个元素设为枢轴,从左边找一个大于枢轴的元素,从右边找一个小于枢轴的元素,进行交换;最终小于枢轴的元素在枢轴的左边,大于枢轴的元素在枢轴的右边;对分开的两部分再次使用快速排序进行排序。

\textbf{示例程序:}排序范围为[start, end]

\begin{lstlisting}
void quicksort(int a[], int start, int end) {
	if (start < end) {	// 递归终止条件
		int i = start, j = end + 1;
		int x = a[start];
		while (true) {
			// 从左侧找到一个大于枢轴的元素
			while (a[++i] < x && i < end);
			// 从右侧找到一个小于枢轴的元素
			while (a[--j] > x && j > start);
			// 迭代器相遇退出循环
			if (i >= j)
				break;
			// 交换两个元素
			swap(a[i], a[j]);
		}
		// 将枢轴归位
		a[start] = a[j];
		a[j] = x;
		// 注:也可使用STL partition函数,注意范围
		// 对左半边排序
		quicksort(a, start, j - 1);
		// 对右半边排序
		quicksort(a, j + 1, end);
	}
}
\end{lstlisting}

\subsection{散列}

\subsubsection{散列的定义与整数散列}

散列(hash)即\textbf{将元素通过一个函数转换为整数,使得该整数可以尽量唯一地代表这个元素}。这个转换函数称为\textbf{散列函数$H$}。对于\textbf{整数$k$},常用的方法有把$k$作为数组下标;也可使用\textbf{除留余数法},即把$k$除以一个数$m$得到的余数作为hash值的方法。这样就可以把很大的数转换为不超过$m$的整数。

一般来说,可以使用STL map/unordered\_map来直接使用hash的功能,因此除非必须模拟这些方法或是对算法效率要求高,一般不需要自己实现。

\subsubsection{字符串散列}

若$P(x,y),0\leq x,y\leq m$表示二维平面上的一个点,将其映射为一个整数的散列函数可以写为$H(P)=x*m+y$。

对于字符串,将其映射为一个整数,使该整数可以唯一地代表字符串。对于只有大写字母的字符串,可以将其当做26进制的数,然后将其转换为十进制,$H[i]=H[i-1]*26+\textrm{index}(\textrm{str}[i]))$。当字符串较长时,产生的整数会非常大。为了应对这种情况,可以将产生的结果对一个整数$m$取模,即$H[i]=(H[i-1]*26+\textrm{index}(\textrm{str}[i]))\%m$,此方法可能会导致冲突。

在int数据范围内,如果把进制数设置为一个$10^7$级别的素数$p$(如$10^7+19$),同时把$m$设置为一个$10^9$级别的素数(如$10^9+7$),产生冲突的概率非常小$H[i]=(H[i-1]*p+\textrm{index}(\textrm{str}[i]))\%m$。

例如:给出$n$个只有小写字母的字符串,求其中不同的字符串个数。

对于这个问题,如果只用字符串hash来做,只需要将$n$个字符串的hash求出来,去重排序即可,当然也可以用set或者map直接做,速度比字符串hash慢一些。

\begin{lstlisting}
const int MOD = 1e9 + 7;
const int P = 1e7 + 19;
vector<int> ans;

long long hash(string str) {
	long long h = 0;
	for (int i = 0; i < str.length(); ++i) 
		h = (h * P + str[i] - 'a') % MOD;
	return h;
}

int main() {
	long long id;
	string str;
	while(getline(cin, str), str != '#') {
		id = hash(str);
		ans.push_back(id);
	}
	sort(ans.begin(), ans.end());	// 排序
	int count = 0;
	for (int i = 0; i < ans.size(); ++i)
		if (i == 0 || ans[i] != ans[i - 1])
			count++;	// 统计不同的数的个数
	printf("%d", count);
	return 0;
}
\end{lstlisting}

\subsection{贪心}

\subsubsection{简单贪心}

\textbf{贪心算法:当前最好选择。}一般来说,\textbf{如果在想到某个似乎可行的策略,并且自己无法举出反例,那么就勇敢去实现它}。

例题:给定数字0-9若干个,可以任意排列这些数字,全部使用。使得到的数字最小,0不能做首位。

\begin{lstlisting}
int count[10] = {0};
string str;
getline(cin, str);
for (auto i = str.begin(); i < str.end(); ++i)
	count[(*i) - '0']++;	// 统计数字个数
for (int i = 1; i < 10; ++i)	// 输出第一个数字
	if (count[i] > 0) {
		printf("%d", i);
		count[i]--;
		break;
	}
for (int i = 0; i < 10; ++i)	// 依次输出剩余数字
	for (int j = 0; j < count[i]; ++j)
		printf("%d", i);
\end{lstlisting}

\subsubsection{区间贪心}

给出一个\textbf{区间不相交问题}:\textbf{给出$n$个开区间$(x,y)$,从中选择尽可能多的开区间,使得这些开区间两两没有交集}。例如对于开区间$(1,3),(2,4),(3,5),(6,7)$来说,可以选出最多三个区间$(1,3),(3,5),(6,7)$,它们互相没有交集。

首先考虑最简单的情况,若开区间$I_1$被开区间$I_2$包含,显然选择$I_1$是最好,如果选择$I_1$,则有更大的空间去容纳其他的开区间。

接下来把所有区间按左端点$x$从大到小排序,如果去掉区间包含的情况,那么一定有$y_1>y_2>...>y_n$成立。此时最右边有一段是一定不会和其他区间重叠的,此时应该\textbf{总是先选择左端点最大的区间}。

\begin{lstlisting}
struct Interval {
	int x, y;
} I[1000];

bool cmp(Interval a, Interval b) {
	if (a.x != b.x)	// 先按左端点从大到小排序
		return a.x > b.x;
	else	// 左端点相同的按右端点从小到大排序
		return a.y < b.y;
}

int n, i;
scanf("%d", &n);
for (i = 0; i < n; ++i)
	scanf("%d%d", &I[i].x, &I[i].y);
sort(I, I + n, cmp);	// 把区间排序
// ans 记录不相交个数, tx记录上一个被选中区间左端点
int ans = 1, tx = I[0].x;
for (i = 1; i < n; ++i)
	if (I[i].y <= tx) {	// 若该区间右端点在 tx 左边
		tx = I[i].x;	// 以 I[i] 作为新选中区间
		ans++;	// 不相交区间个数 +1
	}
printf("%d", ans);	// 输出结果
\end{lstlisting}

\subsection{二分法}

\subsubsection{二分查找}

对于\textbf{有序容器},可以使用STL中的binary\_search函数进行二分法查找。

\begin{lstlisting}
// 函数原型,查找范围 [first, last)
bool binary_search(it first, it last, T &val);
// 例如
int a[] = {1, 2, 3, 4, 5, 6};
bool have5 = binary_search(a, a + 6, 5);
\end{lstlisting}

\subsubsection{快速幂}

给定三个整数$a, b, m(a<10^9, b<10^6, 1<m<10^9)$,求$a^b\%m$。在快速幂的算法中,若$b$是奇数,$a^b=a*a^{b-1}$;若$b$是偶数,$\displaystyle{ a^b=a^{b/2}*a^{b/2}}$。

针对不同的题目,需要注意的是:(1)若初始时$a\geq m$,需要在进入函数前让$a$对$m$取模;(2)若$m=1$,可以直接在函数外部判为0。

快速幂的递归写法不做讨论。在迭代写法中,如果把$b$写成二进制,则$b$就可以写成若干二次幂的和,即$a^b=a^{2^k+...+4+2+1}=a^{2^k}*...*a^4*a^2*a$,若$b$的二进制的$i$号位为1,那么项$a^{2^i}$就被选中。

\begin{lstlisting}
typedef long long ll;
ll bin_pow(ll a, ll b, ll m) {
	if (m == 1)
		return 0;
	if (a >= m)
		a %= m;
	ll ans = 1;
	while (b > 0) {
		if (b % 2 == 1)		// 若 b 的二进制末尾为 1
			ans = ans * a % m;	// 令 ans 累积上 a
		a = a * a % m;	// 令 a 平方
		b = b / 2;	// 将 b 的二进制右移一位
	}
	return ans;
}
\end{lstlisting}

\subsubsection{lower\_bound, upper\_bound}

在\textbf{有序容器}中,lower\_bound(first, last, val)函数用来寻找在容器[first, last)范围内\textbf{第一个值大于或等于val}的元素位置;upper\_bound(first, last, val)函数用来寻找在容器[first, last)范围内\textbf{第一个值大于val}的元素位置。两个函数的复杂度均为$O(\log(\textrm{last}-\textrm{first}))$。
