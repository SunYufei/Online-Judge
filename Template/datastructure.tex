\subsection{数组}

\subsubsection{普通数组}

\begin{lstlisting}[language=python]
l = ['a', 'b', 3]
# 长 1000,初始值全为 0
l = [0] * 1000
# 100 * 100 大小初始值为 0 的二维数组
l = [[0] * 100 for _ in range(100)]
\end{lstlisting}

\begin{lstlisting}[language=c++]
// 全部赋值 0
int a[10] = {0};
// 部分指定初始值,其余为默认值
int a[10] = {5, 3, 2, 4, 1};
// 二维数组初始化
int a[5][6] = {{3, 1, 2}, {8, 4}, {}};
// 字符数组可以使用字符串初始化,仅限初始化
char str[15] = "Hello, World!";
\end{lstlisting}

\subsubsection{赋值函数memset与fill}

使用memset与fill函数可以对数组赋相同的值。memset对每个字节赋同样的值,\textbf{只建议赋值0和$-$1},fill函数可以给数组赋任意值,memset执行速度快。

\begin{lstlisting}[language=c++]
int a[10];
// memset(数组名, 值, sizeof(数组名));
memset(a, 0, sizeof(a));
// 将 [start, end) 之间的元素赋值为 value
// fill(iterator start, it end, const T& val);
// 将从 start 开始的 n 个元素赋值为 val
// fill_n(iterator start, Size n, const T& val);
fill(a, a + 10, 2);
fill(a, 10, 3);
\end{lstlisting}

\subsubsection{变长数组}

变长数组常用于解决普通数组超内存的情况,也可用于以邻接表方式储存图。

Python提供了变长的连续存储数据结构list,可以进行索引、切片、加、乘、检查成员等操作:

\begin{lstlisting}[language=python]
a = ['a', 'b', 3]	# 初始化
# 下标访问
a[0], a[1], a[-1]	# 'a','b', 3
# 切片,注意下标范围 [start, end)
a[1 : 2]	# ['b']
a.append(x)	# 将 x 插入到 a 尾部
t = a.pop(0)	# 剔除 0 号元素并返回 0 号元素值
a.reverse()	# 列表倒置
# 合并(不去重)
a = [1, 2, 3], b = [3, 4, 5]
a + b	# [1, 2, 3, 3, 4, 5]
\end{lstlisting}

STL中提供了变长数组vector:

\begin{lstlisting}[language=c++]
vector<TYPE> v;	// 一维数组
vector<vector<TYPE> > vv;	// 二维数组
vector<TYPE> v[size];	// 一维定长,另一维边长数组
v.push_back(x);	// 将 x 插到 vector 尾部,O(1)
v.pop_back();	// 将 vector 尾部元素删除,O(1)
size_type t = v.size();	// vector 长度,O(1)
v.clear();	// 清空 vector,O(n)
v.insert(it, x);	// 在迭代器 it 处插入 x,O(n)
v.erase(it);	// 删除迭代器 it 处元素,O(1)
v.erase(begin, end); // 删除 [begin, end) 内元素
\end{lstlisting}

\subsection{栈}

栈是一种\textbf{后进先出}的数据结构,常用于模拟实现递归,防止程序对栈内存的限制导致程序出错。一般的递归层数达到几千或几万层就会崩溃,使用栈模拟递归算法可以避免此类问题出现。

Python LifoQueue是一个后进先出的队列,可以当做栈来使用:\label{lifoqueue}

\begin{lstlisting}[language=python]
from queue import LifoQueue

# maxsize 设置最大长度,0 为无限长
q = LifoQueue(maxsize=0)
q.put(x)	# 将 x 入队列
item = q.get()	# 获取队首元素
is_empty = q.empty()	# 判断队列是否为空
is_full = q.full()	# 判断队列是否已满
s = q.qsize()	# 返回队列大小(多线程不可靠)
\end{lstlisting}

STL stack初始化及常用函数:

\begin{lstlisting}[language=c++]
stack<TYPE> s;
s.push(const TYPE &val);		// 将 val 入栈,O(1)
s.pop();	// 弹出栈顶元素,O(1)
s.clear();	// 清除所有元素,O(n)
TYPE t = s.top();	// 获得栈顶元素,O(1)
bool e = s.empty();	// 检测栈是否为空,O(1)
size_type t = s.size();	// 栈内元素个数,O(1)
\end{lstlisting}

\subsection{队列}

队列是一种\textbf{先进先出}的数据结构。当需要实现广度优先搜索时,可以使用queue代替手动实现的队列,提高程序准确性。

Python  Queue是一个先进先出的队列,其常用函数与\ref{lifoqueue} LifoQueue相同。

Python deque是双端队列,使用时注意插入删除操作的位置及相关函数。

\begin{lstlisting}[language=python]
from collections import deque

# deque 可从 iterable 指定的数据创建
# maxlen 指定最大长度
d = deque([iterable[, maxlen]]])
d.append(x)		# 添加 x 到右端
d.appendleft(x)	# 添加 x 到左端
d.clear()	# 移除所有元素
c = d.count(x)	# 计算 deque 中 x 元素个数
d.extend(iterable)	# 扩展 deque 的右侧
d.extendleft(iterable) # 扩展 deque 的左侧
item = d.pop()	# 移除并返回最右侧元素
item = d.popleft()	# 移除并返回最左侧元素
d.remove(x)	# 找到第一个 x 并移除,未找到抛出异常
d.rotate(n = 1)	# 向右循环 n 步,若 n 为负向左循环
\end{lstlisting}

STL queue初始化及常用函数:

\begin{lstlisting}[language=c++]
queue<TYPE> q;
q.push(const TYPE &val);	// 将 val 入队列,O(1)
q.pop();	// 将队首元素出队,O(1)
bool e = q.empty();	// 检测队列是否为空,O(1)
size_type t = q.size();	// 队列元素个数,O(1)
TYPE &back();	// 返回队尾元素的引用,O(1)
TYPE &front();	// 返回队首元素的引用,O(1)
\end{lstlisting}

\textbf{注意:}在使用front和pop函数前,必须用empty判断队列是否为空。

\subsection{优先队列}

优先队列底层用堆来实现。在优先队列中,\textbf{队首元素}是队列中\textbf{优先级最高}的一个。优先队列可以解决一些贪心问题,也可用于对Dijkstra算法进行优化。

Python PriorityQueue初始化及常用函数:

\begin{lstlisting}[language=python]
from queue import PriorityQueue

q = PriorityQueue(maxsize=0)	# maxsize 最大长度
q.put((priority_number, data))	# 插入元素
data = q.get()	# 获取队首元素
is_empty = q.empty()	# 队列是否为空
# 当队列的元素是自定义时,需要在元素类中定义比较规则
\end{lstlisting}

STL priority\_queue初始化及常用函数:

\begin{lstlisting}[language=c++]
priority_queue<TYPE> q;
// 只能通过 top 函数访问队首元素,使用前判断是否为空
TYPE t = q.top();
q.push(x);	// 将 x 入队列
q.pop();	// 将队首元素出队列
bool t = q.empty();	// 判断队列是否为空
size_type s = q.size();	// 队列元素个数
\end{lstlisting}

定义优先队列内元素的优先级时,基本数据类型可以使用less或greater来指定优先级,其中less表示大的优先,greater表示小的优先;结构体元素需要重载小于号运算符。

\begin{lstlisting}[language=c++]
// 第二个参数指定存储容器
priority_queue<int, vector<int>, less<int>> q;
struct fruit {
	string name;
	int price;
	friend bool operator< (fruit f1, fruit f2) {
		return f1.price < f2.price;
	}
}
priority_queue<fruit> q;	// 自定义元素类型
\end{lstlisting}

\subsection{字符串}

\subsubsection{C string.h}

string.h中包含许多用于字符数组的函数:

\begin{lstlisting}[language=c++]
int strlen(char* s);	// 字符串长度
// 字符串比较,原则是字典序
// 相等 0;s1 > s2,正数;s1 < s2,负数
int strcmp(char* s1, char* s2);
void strcpy(char* s1, char* s2);	// 将s2复制给s1
void strcat(char* s1, char* s2);	// 将s2加到s1后
\end{lstlisting}

\subsubsection{STL string}

STL string对字符串常用需求进行了封装:

\begin{lstlisting}[language=c++]
string str = "abcd";
str += "ef";	// 将 "ef" 添加到 str 尾部
// == != < <= > >= 字符串比较,规则是字典序
int len = str.length();	// 字符串长度,O(1)
/* insert 函数,O(n) */
str.insert(pos, s);	// 在 pos 位置插入字符串 s
// 在 it 插入串 [first, last)
str.insert(it, first, last);
/* erase 函数,O(n) */
str.erase(it);	// 删除 it 指向元素
str.erase(first, last);	// 删除区间[first, last)
str.erase(pos, len); // 从pos开始删除 len 个字符
str.clear();	// 清空字符串,O(1)
str.substr(pos, len);	// 从 pos 开始长 len 的子串
string::npos	// 常数,用于 find 函数匹配失败返回值
/* find 函数 */
int p = str.find(s);	// s 在 str 的位置
int p = str.find(s, pos);	// 从str的pos位匹配s
/* replace 函数 */
// 把 str 从 pos 开始,长 len 的子串替换为 s
string s2 = str.replace(pos, len, s);
// 把 str 的 [first, last) 范围的子串替换为 s
string s2 = str.replace(first, last, s);
char* s = str.c_str();	// 转换为字符数组
\end{lstlisting}


\subsection{集合}

Python的集合set是一个\textbf{无序的}、\textbf{不含重复元素}的容器,不记录元素插入点或位置,不可通过下标访问,查询性能相比list高很多。

\begin{lstlisting}[language=python]
s = set()
s.add(x)	# 将 x 插入集合中,若 x 存在,不进行操作
s.update(x)	# 更新 s,加上 x 中的元素
s.remove(x)	# 将 x 从集合中移除,若 x 不存在报错
s.discard(x)	# 将 x 从集合中移除,允许 x 不存在
item = s.pop()	# 随机删除一个元素
s.clear()	# 清空集合
s.issubset(x)	# s 是否为 x 的子集
x & y	# 交
x | y	# 并
x - y	# 差
x ^ y	# 对称差,即 x 和 y 的交集减并集
\end{lstlisting}

STL集合set是一个\textbf{内部自动有序}且\textbf{不含重复元素}的容器,最重要的作用是自动去重并按升序排序。

\begin{lstlisting}[language=c++]
set<TYPE> s;
set<TYPE>::iterator it;	// 迭代器
// 遍历
for(auto it = s.begin(); it != s.end(); ++it)
do_something(*it);
s.insert(x);	// 将 x 插入集合中,O(log n)
it = s.find(x);	// 返回 x 的迭代器,O(log n)
// 删除迭代器 it 指向元素,O(1),可与 find 结合
s.erase(it);
s.erase(x);	// 删除值为 x 的元素,O(log n)
s.erase(begin, end); // 删除 [begin, end) 内元素
size_type t = s.size();	// set 内元素个数,O(1) 
s.clear();	// 清空所有元素,O(n)
\end{lstlisting}

C++ 11中增加了unordered\_set,以散列代替set内部红黑树,去重但不排序,速度相比set快很多。

\subsection{映射}

通过名字来引用值的数据结构称为映射,名字成为键key,与值value成对。常用于:建立字符(串)与整数之间映射;判断大整数或者其他类型数据是否存在;建立字符串与字符串之间的映射。

Python dict字典是键值对的无序可变集合,具有极快的查找速度。

\begin{lstlisting}[language=python]
d = dict()	# 空字典或 d = {}
d = {key1: val1, key2: val2}	# 指定初值
d = dict(key1=val1, key2=val2)
len(d)	# 返回字典长度
val = d[key]	# 下标访问
d[key] = value	# 将 d[key] 设置为 value
val = d.get(key)	# get 方法访问
val = d.pop(key)	# 删除 key,返回对应 value
(key, val) = d.popitem()	# 随机删除并返回一对 k-v
d.update((key, val))	# 用指定的(key, val)更新字典
d.items()	# 返回可遍历的 (key, val) 元组数组
d.keys()	# 返回字典所有键
key in d	# 判断 key 是否字典中
d.values()	# 返回字典所有值
del d[key]	# 删除 key 对应的键值对,不存在抛出异常
d.clear()	# 清空字典
\end{lstlisting}

STL map可以将任何类型或容器映射到任何类型或容器。map中\textbf{键值唯一},\textbf{插入时按键值排序}。map中存放的是pair结构体,有first和second两个不同类型的成员。



\begin{lstlisting}[language=c++]
map<key_type, value_type> mp;
mp['a'] = 1;	// 下标访问
// 迭代器访问
for (auto i = mp.begin(); i != mp.end(); ++i)
do_something(i->first, i->second);
it = mp.find(key);	// 返回键为 key 的迭代器
mp.erase(it);	// 删除 it 指向元素
mp.erase(key);	// 删除键值为 key 的元素
mp.erase(first, last);	// 删除[first, last)内元素
size_type s = mp.size();	// 映射个数
mp.clear();	// 清空所有元素
\end{lstlisting}

C++ 11提供unordered\_map,以散列代替红黑树实现map,只处理映射不排序,速度比map快得多。
