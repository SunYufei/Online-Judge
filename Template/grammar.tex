\subsection{C++头文件引入技巧}

使用C++编写程序时只需引入如下头文件,包含STL、C/C++ I/O、math.h、string.h等。

\begin{lstlisting}[language=c++]
#include <bits/stdc++.h>
using namespace std;
\end{lstlisting}

\subsection{C++格式化输入输出}

对于C语言中的输入输出函数

\begin{lstlisting}[language=c]
scanf("格式化字符串", <参量表>);
printf("格式化字符串", <参量表>);
\end{lstlisting}

格式化字符串中的符号规则如下表所示:

\begin{table}[htbp]
	\centering
	\begin{tabular}{cc}
		\hline
		\textbf{符号} & \textbf{说~~明} \\
		\hline
		\hline
		\% & 格式说明起始符号,不可缺少 \\
		\hline
		- & 有:左对齐输出;无:右对齐输出 \\
		\hline
		0 & 有:指定空位为0;无:指定空位为空格 \\
		\hline
		\multirow{2}{*}{$m.n$} & $m$:输出项在输出设备上所占字符数 \\
		& $n$:精度,实数的小数位数,默认$n=6$ \\
		\hline
		l & int$\rightarrow$long;float$\rightarrow$double \\
		\hline
		h & int$\rightarrow$short \\
		\hline
		\hline
		\%d & 十进制int \\
		\hline
		\%ld & 十进制long \\
		\hline
		\%lld & 十进制long long \\
		\hline
		\%u & 十进制unsigned int \\
		\hline
		\%o & 八进制unsigned int \\
		\hline
		\%x & 十六进制unsigned int (\%X 大写) \\
		\hline
		\%f & float\\
		\hline
		\%lf & double\\
		\hline
		\%e & 指数\\
		\hline
		\multirow{2}{*}{\%g} & 保证$n$(默认6)位有效数字的情况下 \\
		&使用小数方式,否则使用科学计数法 \\
		\hline
		\hline
		\%c & 一个字符char \\
		\hline
		\%s & 字符串char * \\
		\hline
	\end{tabular}
\end{table}

cin与cout是C++中的输入输出对象,性能相比scanf与printf非常差,推荐在输入输出时使用scanf与printf,只有在必要时才使用cin和cout。

例如,STL string的读写只能用cin和cout:

\begin{lstlisting}[language=c++]
string str;	
cin >> str;
cout << str;
getline(cin, str);	// 读一整行使用 getline 函数
\end{lstlisting}

\subsection{Python语言输入输出技巧}

Python语言的输入技巧:

\begin{lstlisting}[language=python]
# 一个整数
n = int(input())

# 一个字符串
s = input()

# 两个整数,用空格分开
n, m = map(int, input().split())

# 一行整数,空格分开
l = list(map(int, input().split()))
l = [int(v) for v in input().split()]
s = set(map(int, input().split()))
s = {int(v) for v in input().split()}

# 一些字符串,空格分开
l = input().split()
\end{lstlisting}

Python的格式化输出语法与C语言中的printf语法相似。

\begin{lstlisting}[language=python]
print("格式化字符串" % (<参量表>))
print("%d%02d" % (hour, minute))

# 在不方便指定参量类型时,推荐使用 format
print("格式化字符串".format(<参量表>))
print("{0:d}{1:02d}".format(hour, minute))
\end{lstlisting}

Python输出时需要注意的一些问题:

\begin{lstlisting}[language=python]
# 按序输出 list 中的元素,空格分开结尾无空格
l = [1, 2, 3, 4, 5]
print(' '.join(map(str, l)))	# 注意此处的 str
print(' '.join([str(v) for v in l]))

# 整数输出时注意末尾的.0
a = 5 / 5, b = 6 // 5
print(a, int(a), b)	# 1.0 1 1
\end{lstlisting}

\subsection{算术运算符}

C++与Python的$+$、$-$、$*$、$/$及$\%$运算符相同;对于int类型的除法,C++返回结果整数部分,Python得到含小数部分的结果,若想与C++一样只得到整数部分,需要使用“$//$”运算符。Python中没有自增自减运算符“$++$”和“$--$”。

\subsection{常用数学函数}
\label{mathfunc}

\begin{table}[htbp]
	\centering
	\begin{tabular}{ccc}
		\hline
		& C++ & Python \\
		& \#include<cmath> & import math \\
		\hline
		\multirow{2}{*}{$|x|$} & abs(int x) & abs(x) \\
		& fabs(double x) & math.fabs(x) \\
		\hline
		$\lceil x \rceil$ & ceil(double x) & math.ceil(x) \\
		$\lfloor x \rfloor$ & floor(double x) & math.floor(x) \\
		\hline
		$r^p$ & pow(double r, p) & math.pow(r, p) \\
		$e^x$ & exp(double x) & math.exp(x) \\
		\hline
		$\sqrt{x}$ & sqrt(double x) & math.sqrt(x) \\
		\hline
		$x!$ & / & math.factorial(x) \\
		\hline
		% $\log_2 x$ & / & math.log2(x) \\
		$\log_{e} x$ & log(double x) & math.log(x) \\
		$\log_{10} x$ & log10(double x) & math.log10(x) \\
		$\log_{a} x$ & / & math.log(x, a) \\
		\hline
		$\sin x$ & sin(double x) & math.sin(x) \\
		$\cos x$ & cos(double x) & math.cos(x) \\
		$\tan x$ & tan(double x) & math.tan(x) \\
		\hline
		$\arcsin x$ & asin(double x) & math.asin(x) \\
		$\arccos x$ & acos(double x) & math.acos(x) \\
		$\arctan x$ & atan(double x) & math.atan(x) \\
		\hline
		rad $\to\deg$ & / & math.degrees(x) \\
		$\deg\to$ rad & / & math.radians(x) \\
		\hline
		round & round(double x) & round(x, n) \\
		\hline
		$\gcd (a, b)$ & 见\ref{gcd} & math.gcd(a, b) \\
		\hline
		$\pi$ & M\_PI& math.pi \\
		$e$ & M\_E & math.e \\
		\hline
		\multirow{3}{*}{$\infty$} & \#include<limits> & \multirow{3}{*}{math.inf} \\
		& INT\_MAX & \\
		& UINT\_MAX 等 & \\
		\hline
	\end{tabular}
\end{table}

\textbf{注意:}三角函数中的$x$为弧度制。

\subsection{Python 表达式求值}

Python的eval函数可以将str当成有效的表达式并求值返回结果,可以与math结合在表达式求值题目中发挥巨大的作用。例如:

\begin{lstlisting}[language=python]
expession = '2 + 3 * 5 - 6 // 4'
print(eval(expression))	# 16
\end{lstlisting}

也可以进行其他表达式计算,如CSP 201709-3 JSON查询。

\begin{lstlisting}[language=python]
t = eval('obj' + ''.join(['[' + repr(x) + ']' for x in c]))
\end{lstlisting}

repr函数将对象转化为供解释器读取的形式。

\subsection{浮点数比较}

浮点数在计算机中的存储不总是精确地,在经过大量计算后,浮点数的存储会产生一定的误差,在使用“==”运算符进行比较时会出现差错。此时需要引入一个极小数eps来对误差进行修正。

\begin{lstlisting}[language=c++]
const double eps = 1e-8;
fabs(a - b) < eps;	// 相等
a - b > eps;	// a > b
a - b < -eps;	// a < b
\end{lstlisting}
